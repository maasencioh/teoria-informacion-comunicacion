\documentclass[twocolumn]{article}
\usepackage{graphicx}
\usepackage{url}
\usepackage[spanish]{babel}
\selectlanguage{spanish}
\usepackage[utf8]{inputenc}
\usepackage{amsmath}

\title{Primer trabajo teoría de la información y la comunicación}
\author{Miguel~Angel~Asencio~Hurtado, Ana~María~Rodríguez~Reyes}

\begin{document}
\maketitle
\textbf{1. Desarrollar en series de Fourier}

$$f(t) = t^2,\; -\pi \leq t \leq \pi$$

\textbf{2. Desarrollar en series de Fourier}

$$f(t) = t \, sin(t),\; -\pi \leq t \leq \pi$$

\textbf{3. Desarrollar en series de Fourier}

$$f(t) = t,\; -\pi \leq t \leq \pi$$

\textbf{4. Desarrollar en series de Fourier}

$$f(t) = \begin{cases}
\pi +t, &-\pi \leq t \leq 0\\
t, &0 \leq t \leq \pi
\end{cases}$$

\textbf{5. Hallar el periodo}

\textbf{a)} $f(t) = sin(\frac{2\pi}{b-a})t$

\textbf{b)} $f(t) = sin(t) + \frac{1}{3}sin(3t) + \frac{1}{5}sin(5t)$

\textbf{c)} $f(t) = cos(10t) + cos((10+\pi)t)$

\textbf{6. Definición del proyecto}

\textit{Estado del arte}

\textit{Desarrollo}


\end{document}
